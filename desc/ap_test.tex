\documentclass{utap}

\usepackage{fancyvrb}
\usepackage{amsmath}
\usepackage{xepersian}

\title{تمرین آزمون نرم‌افزار}
\author{\href{mailto:hadi.safari@ut.ac.ir?subject=[AP\%20S98 UT]\%20}{هادی صفری}}
\course{برنامه‌سازی پیشرفته}
\lecturer{رامتین خسروی}
\deadline{?، ساعت ۲۳:۵۵}

\begin{document}
    \maketitle

    \section{مقدمه‌ای بر آزمون نرم‌افزار}

    با آزمون نرم‌افزار و \textit{آزمون واحد}\LTRfootnote{unit testing} و مزایا و لزوم آن‌ها در کلاس درس آشنا شده‌اید. آزمون واحد روشی  برای آزمودن واحدهای کد منبع و اطمینان از مناسب~بودن آن‌ها است. واحد کوچک‌ترین بخش آزمودنی کد منبع است که در زبان‌ها و الگوهای برنامه‌نویسی‌ای که با آن‌ها آشنایی دارید معمولاً معادل یک تابع است.
    واحد تحت تست\LTRfootnote{unit under test} معمولاً یا تست‌ها را می‌گذراند\LTRfootnote{pass} یا در آن‌ها شکست می‌خورد\LTRfootnote{fail}. گاهی ممکن است آزمونی در حلقهٔ بینهایت بیفتد یا زمان آن تمام شود\LTRfootnote{time out}.
    برای اطلاعات بیشتر دربارهٔ آزمون واحد می‌توانید به منبع \cite{wikibooks18} مراجعه کنید.

    \subsection{طراحی آزمون‌ها}

    آزمون واحد معمولاً بر اساس ادعاها\LTRfootnote{assert} بنا می‌شود. انتظار می‌رود ادعاها برقرار باشند و در غیر این صورت آزمون شکست می‌خورد.
    یک راه معمول طراحی آزمون واحد آن است که ابتدا شرایط مناسب برای آزمون فراهم می‌گردد؛ مثلاً یک شیء از کلاس تحت آزمون با شرایط مورد نظر ایجاد شود. سپس عملیات خاصی صورت می‌گیرد که هدف آزمون بررسی صحت آن عملیات خاص است؛ مثلاً یکی از توابع کلاس تحت آزمون صدا می‌شود تا محاسبه‌ای را انجام دهد و نتیجهٔ آن ذخیره می‌گردد. در انتها با بازبینی اتفاقات رخ‌داده از صحت عملیات انجام‌شده اطمینان حاصل می‌شود؛ مثلاً نتیجهٔ اجرای تابع محاسباتی مذکور را با مقدار مورد انتظار مقایسه می‌کنیم. این الگو به الگوی \lr{AAA}\LTRfootnote{Arrange, Act, Assert} مشهور است \cite{dietrich14,vocke18}. تعدادی آزمون مختلف بسته به نیاز طراحی و دسته‌بندی می‌شود و نهایتاً یک برنامه یا ابزار آن‌ها را اجرا می‌کند.

    \par در طراحی آزمون واحد خوب است به برخی نکات توجه داشته باشید \cite{dietrich14}:
    \begin{itemize}[nosep]
        \item آزمون واحد باید تا حد امکان ساده باشد و منطق پیچیده‌ای نداشته باشد.
        \item هر آزمون واحد باید دقیقاً یک شرط را بیازماید. در صورت شکست یک آزمون واحد باید بتوان به~راحتی متوجه شد کدام عملکرد دچار مشکل است.
        % \item گاهی برای محدود~کردن آزمون واحد به یک شرط گاهی باید وابستگی کلاس‌ها یا توابع به کلاس‌ها یا توابع دیگر را از بین برد. برای این کار از انواع \textit{بدل‌های آزمونی}\LTRfootnote{test doubles} (\textit{اشیاء بی‌فایده}\LTRfootnote{dummy objects}، \textit{اشیاء جعلی}\LTRfootnote{fake objects}، \textit{جاسوس آزمونی}\LTRfootnote{test spy}، \textit{باقیماندهٔ آزمونی}\LTRfootnote{test stubs} و \textit{اشیاء مقلد}\LTRfootnote{mock objects}) استفاده می‌شود.
        % منبع \cite{vocke18} توضیحات بیشتری دربارهٔ  بدل‌های آزمونی در بر دارد.
        \item آزمون‌ها نباید اثرات جانبی داشته باشند. اگر آزمون‌ها اثر جانبی نداشته باشند تکرارپذیر می‌شوند و ترتیب اجرایشان نیز بی‌اهمیت می‌گردد.
        \item آزمون واحد باید رفتار قابل مشاهده را بیازماید، نه ساختار داخلی کد را. اجزای خصوصی\LTRfootnote{private} کلاس‌ها نیز معمولاً به~عنوان جزئیات پیاده‌سازی در نظر گرفته می‌شوند.
        هرچند آزمون واحد باید کد را به~خوبی پوشش دهد، آزمون‌ها نباید بیش از حد به جزئیات پیاده‌سازی گره خورده باشند تا بیش از حد شکننده نشوند. به جای فکر کردن دربارهٔ  این که «آیا اگر مقادیر \lr{\texttt{x}} و \lr{\texttt{y}} وارد شوند، تابع ابتدا تابع \lr{\texttt{a}} و سپس تابع \lr{\texttt{b}} را صدا خواهد کرد و سپس مجموع نتایج را به~عنوان نتیجهٔ نهایی باز می‌گرداند؟» به این فکر کنید که «آیا اگر مقادیر \lr{\texttt{x}} و \lr{\texttt{y}} وارد شوند، نتیجه برابر \lr{\texttt{z}} خواهد بود؟» \cite{vocke18}.\par
        طراحی موارد آزمون\LTRfootnote{test case} با پیاده‌سازی آزمون‌ها متفاوت است. در بسیاری از روش‌های انتخاب موارد آزمون مناسب، هرچند نه در همهٔ آن‌ها، به ساختار داخلی کد نیز توجه می‌شود تا موارد آزمون بحرانی و شرایط مرزی در نظر گرفته شوند و کد موجود به~خوبی پوشش داده شود؛ اما با تغییر پیاده‌سازی داخلی بدون تغییر رفتار متد یا کلاس، آزمون‌های قبلی نباید شکست بخورند.
        \item کد اصلی باید فاقد منطق مربوط به تست باشد. استفاده از جملاتی مانند \lr{\lstinline{if (TEST_MODE)}} پیشنهاد نمی‌شود.\par
        در صورت نیاز به دسترسی یا تغییر فیلدهای درونی کلاس‌ها می‌توانید از روش‌هایی مانند \textit{بدل‌های آزمونی}\LTRfootnote{test doubles} یا تعریف کلاس‌های مشتق کمک بگیرید.
        \item کد آزمون نیز نوعی کد است و تمام مسائل مربوط به نوشتن کد تمیز دربارهٔ آن صادق است.
    \end{itemize}

    \subsection{آزمون در \lr{C++}}

    هرچند می‌توان با استفاده از ماکروی\LTRfootnote{macro} \lr{\lstinline{assert}} که در فایل سرآیند\LTRfootnote{header file} \lr{\path{cassert}} (و \lr{\path{assert.h}} در \lr{C}) قرار دارد آزمون طراحی کرد، معمولاً برای نوشتن آزمون از چهارچوب‌های خاصی استفاده می‌شود تا نوشتن آزمون‌های پیچیده ساده‌تر شود. در \lr{C++} نیز چنین چهارچوب‌هایی وجود دارد؛ در حال حاضر دو چارچوب \lr{\href{https://github.com/abseil/googletest}{Google~Test}} و \lr{\href{https://www.boost.org/doc/libs/1_68_0/libs/test/}{Boost.Test}} رایج‌ترند.

    چهارچوب \lr{Catch} و نسخهٔ دوم آن (\lr{\href{https://github.com/catchorg/Catch2}{Catch2}}) امکانات زیادی فراهم می‌کند و استفاده از آن نیز نسبتاً آسان‌تر است.
    به دلایل مذکور، این کتابخانه برای استفاده انتخاب شد.
    برای آشنایی با این کتابخانه می‌توانید به مخزن گیت‌هاب آن که در آدرس \lr{\url{https://github.com/catchorg/Catch2}} در دسترس است مراجعه کنید.
    مطالعهٔ مستندات پروژه از جمله صفحات \href{https://github.com/catchorg/Catch2/blob/master/docs/tutorial.md}{آموزش} مقدماتی و \href{https://github.com/catchorg/Catch2/blob/master/docs/assertions.md}{ادعاها} برای آشنایی با کتابخانه و نحوهٔ استفاده از آن پیشنهاد می‌گردد.

    \subsubsection{آزمودن کلاس \lr{\texttt{Person}}}
    در ادامه نحوهٔ نوشتن آزمون برای یک کلاس ساده را بررسی خواهیم کرد.

    فرض کنید کلاسی به نام \lr{\lstinline{Person}} وجود دارد که نام افراد را در خود نگه می‌دارد:
    \begin{latin}\lstinputlisting[firstline=6, lastline=16]{../person/Person.hpp}\end{latin}

    سازندهٔ این کلاس در صورت دریافت \lr{\lstinline{firstname}} خالی استثنای\LTRfootnote{exception} \lr{\lstinline{std::invalid_argument}} پرتاب می‌کند.

    همچنین، این کلاس تابعی به نام \lr{\lstinline{get_fullname}} دارد که قرار است با به هم چسباندن نام و نام خانوادگی افراد نام کامل آن‌ها را برگرداند:
    \begin{latin}\lstinputlisting[firstline=16, lastline=16]{../person/Person.cpp}\end{latin}

    حال کلاس آزمونی به نام \lr{\lstinline{person_test}} برای آزمودن کلاس \lr{\lstinline{Person}} می‌نویسیم:
    \begin{latin}\lstinputlisting[numbers=left,frame=leftline]{../person/person_test.cpp}\end{latin}

    این کلاس شامل تعدادی مورد~آزمون\LTRfootnote{testcase} است که هر یک چیزی را می‌آزمایند. چهارچوب آزمون ‐~به دلیل تعریف~شدن \lr{\lstinline{CATCH_CONFIG_MAIN}}~‐ یک تابع \lr{\lstinline{main}} می‌سازد که همهٔ موارد آزمون را اجرا می‌کند.

    به نحوهٔ آزمون خروجی متدها و همچنین روش استفاده‌شده برای آزمودن استثناها توجه کنید.

    بعد از ترجمه و اجرای همهٔ  کدها مشاهده می‌شود که یکی از آزمون‌ها که مربوط به عملکرد صحیح متد \lr{\lstinline{Person::get_fullname}} است شکست می‌خورد:
    \begin{latin}\footnotesize
    \VerbatimInput{../person/output_failed.txt}
    \end{latin}

    متد مذکور را اصلاح می‌کنیم:
    \begin{latin}\lstinputlisting[firstline=20, lastline=20]{../person/Person.cpp}\end{latin}

    بعد از اصلاح کلاس تحت آزمون، همهٔ آزمون‌ها با موفقیت گذارنده می‌شوند:
    \begin{latin}\footnotesize
    \VerbatimInput{../person/output_passed.txt}
    \end{latin}

    در هنگام نوشتن ادعاها برای انواع مختلف متغیرها به معنی‌دار~بودن و درستی عملگرهای مقایسه‌ای که استفاده می‌کنید دقت کنید. مثلاً دقت کنید در مقایسهٔ انواع عددی ممیز شناور (\lr{\lstinline{float}} و \lr{\lstinline{double}}) عملگر \lr{\lstinline{==}} قابل اتکا نیست و باید از مقایسهٔ قدر مطلق اختلاف دو عدد با یک $\delta$ کوچک استفاده کرد. همچنین هنگام مقایسهٔ اشیاء کلاس‌هایی که خودتان نوشته‌اید باید عملگر مورد استفاده را سربارگذاری کرده باشید.

    \section{تمرین}

    در این تمرین از شما انتظار می‌رود برای چند کلاس یا تابع با محوریت آزمون واحد و با استفاده از چهارچوب \lr{Catch2} آزمون بنویسید.

    \subsection{نحوهٔ طراحی و پیاده‌سازی}

    \subsubsection[کلاس Triangle]{کلاس \lr{\lstinline[basicstyle=\ttfamily]{Triangle}}}

    کلاس \lr{\lstinline{Triangle}} یک مثلث و برخی ویژگی‌های آن را مدل می‌کند.
    در پیاده‌سازی این کلاس که در اختیار شما قرار گرفته است چهار نوع اشتباه متفاوت وجود دارد. برای عملکردهای متفاوت این کلاس آزمون طراحی کنید. انتظار می‌رود در صورت وجود هر یک از این چهار اشکال (یا هر ترکیبی از آن‌ها) حداقل یکی از آزمون‌هایی که نوشته‌اید با شکست مواجه شود و در صورت برطرف~شدن همهٔ اشتباهات پیاده‌سازی آزمون‌ها با موفقیت گذرانده شوند.

    اگر نیاز دارید به متغیرهای داخلی کلاس \lr{\lstinline{Triangle}} دسترسی پیدا کنید می‌توانید کلاسی مانند \lr{\lstinline{TriangleUnderTest}} از آن مشتق کنید و به این کلاس گیرنده‌ها\LTRfootnote{getter} یا متدهای مورد نیاز خودتان را اضافه کنید.

    آزمون‌های خود و همهٔ نیازمندی‌های آن‌ها را \textit{فقط} در یک فایل به نام \lr{\path{triangle_test.cpp}} بنویسید. این فایل در کنار فایل‌های \lr{\path{catch.hpp}} و \lr{\path{Triangle.hpp}} و به همراه نسخه‌های مختلفی از فایل \lr{\path{Triangle.cpp}} ترجمه خواهد شد و آزمون‌های شما ارزیابی خواهد گردید.
    
    \subsubsection[تابع get\_avg\_of\_vector]{تابع \lr{\lstinline[basicstyle=\ttfamily]{get_avg_of_vector}}}

    تابع \lr{\lstinline{get_avg_of_vector}} با دریافت یک بردار\LTRfootnote{vector} از اعداد صحیح، میانگین اعشاری آن‌ها را محاسبه می‌کند.
    علاوه بر پیاده‌سازی اصلی، چند نسخهٔ مختلف از پیاده‌سازی این تابع در اختیار شما قرار گرفته است. آزمون‌هایی برای این تابع طراحی کنید که پیاده‌سازی‌های اشتباه حداقل در یکی از آن‌ها شکست بخورند و پیاده‌سازی‌های درست همه را با موفقیت بگذرانند.

    آزمون‌های خود و همهٔ نیازمندی‌های آن‌ها را \textit{فقط} در یک فایل به نام \lr{\path{get_avg_of_vector_test.cpp}} بنویسید. این فایل در کنار فایل‌های \lr{\path{catch.hpp}} و \lr{\path{get_avg_of_vector.hpp}} و به همراه نسخه‌های مختلف فایل \lr{\path{get_avg_of_vector.cpp}} ترجمه خواهد شد و آزمون‌های شما ارزیابی خواهد گردید.

    \subsubsection[تابع satisfies\_hailstone]{تابع \lr{\lstinline[basicstyle=\ttfamily]{satisfies_hailstone}}\hfill\normalsize\mdseries اختیاری}

    تابع \lr{\lstinline{satisfies_hailstone}} عددی می‌گیرد و بررسی می‌کند آیا یک \textit{عدد تگرگی}\LTRfootnote{Hailstone number}\cite{su} است یا نه.

    یک عدد تگرگی عددی است که با شروع از آن و نوشتن دنباله‌ای با این رابطه، دنباله در نقطه‌ای به عدد ۱ برسد:
    \[ a_n =
      \begin{cases}
        \frac{a_{n - 1}}{2} & \quad \text{اگر $a_{n - 1}$ زوج باشد}\\
        3a_{n - 1} + 1 & \quad \text{اگر $a_{n - 1}$ فرد باشد}\\
      \end{cases}
    \]

    \textit{حدس کولاتز}\LTRfootnote{Collatz conjecture} بیان می‌کند که همهٔ اعداد مثبت تگرگی‌اند.

    سعی کنید جز حالت بدیهی ۰، مورد آزمون دیگری بیابید که تابعی که در اختیارتان قرار گرفته است در آن شکست بخورد.
    اگر این سؤال در زبان \lr{Python} مطرح شده بود می‌توانستید برای آن پاسخی بیابید؟
    آیا می‌توانید کد آزمون زبان \lr{C++} را طوری بنویسید که مستقل از معماری ماشین در حال استفاده نتیجهٔ مورد نظر را تولید کند؟

    \subsection{نحوهٔ تحویل}

    \begin{itemize}
    \item فایل‌های آزمون خود را با نام‌های \lr{\path{AT-SID-triangle_test.cpp}} و \lr{\path{AT-SID-get_avg_of_vector_test.cpp}} در صفحهٔ \lr{CECM} درس بارگذاری کنید که \lr{SID} شمارهٔ دانشجویی شماست؛ برای مثال اگر شمارهٔ دانشجویی شما ۸۱۰۱۹۷۹۹۹ باشد، نام فایل‌های شما باید \lr{\path{AT-810197999-triangle_test.cpp}} و \lr{\path{AT-810197999-get_avg_of_vector_test.cpp}} باشد.

    نیازی به تحویل آزمون تابع \lstinline{satisfies_hailstone} نیست.

    \item آزمون‌های شما باید الزاماً با استفاده از چهارچوب آزمون \lr{Catch2} نوشته شده باشند.
    حتماً با استفاده از \lr{\lstinline{CATCH_CONFIG_MAIN}} اجازه بدهید چهارچوب آزمون خودش تابع \lr{\lstinline{main}} را تولید کند.
    به نمونهٔ بارگذاری‌شده از آزمون کلاس \lr{\lstinline{Person}} دقت کنید.
    
    \item به فرمت و نام فایل‌های خود دقت کنید. از بارگذاری فایل فشرده خودداری کنید.

    \item توجه داشته باشید که با توجه به تست خودکار کدهای شما، عدم رعایت نکات مذکور ممکن است منجر به از~دست~دادن همهٔ نمرهٔ شما بشود.
    
    \item برنامه‌ی شما باید در سیستم‌عامل لینوکس و با مترجم \lr{g++} با استاندارد \lr{C++11} ترجمه و در زمان معقول اجرا شود.
    
    \item هدف این تمرین یادگیری شماست. لطفاً تمرین را خودتان انجام دهید. در صورت کشف تقلب مطابق قوانین درس با آن برخورد خواهد شد.
    \end{itemize}

    \begin{thebibliography}{9}
    \begin{latinitems}
    % \bibitem{alex17} Alex. 2017. Assert and \lstinline{static_assert}. Retrieved from \url{https://www.learncpp.com/cpp-tutorial/7-12a-assert-and-static_assert/}.
    \bibitem{dietrich14} Erik Dietrich. 2014. Introduction to Unit Testing (Don’t Worry, Your Secret is Safe with Me). Retrived from \url{https://daedtech.com/introduction-to-unit-testing-dont-worry-your-secret-is-safe-with-me/}.
    \bibitem{su} Francis E. Su, \textit{et al.} ``Hailstone Numbers.'' In \textit{Math Fun Facts}. Retrived from \url{https://www.math.hmc.edu/funfacts/ffiles/10008.5.shtml}.
    % \bibitem{sumner} Nick Sumner. Exercise: Unit Testing with Google Test. Retrived from \url{http://www.cs.sfu.ca/~wsumner/teaching/exercise-unit-tests.html}.
    \bibitem{vocke18} Ham Vocke. 2018. The Practical Test Pyramid. Retrived from \url{https://martinfowler.com/articles/practical-test-pyramid.html}.
    \bibitem{wikibooks18} Wikibooks. 2018. ``Unit Tests.'' In \textit{Wikibooks, The Free Textbook Project}. Retrieved from \url{https://en.wikibooks.org/wiki/Introduction_to_Software_Engineering/Testing/Unit_Tests}.
    \end{latinitems}
    \end{thebibliography}
\end{document}